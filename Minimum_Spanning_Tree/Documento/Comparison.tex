\documentclass[main.tex]{subfiles}
\begin{document}
\section{Quantum Development Kit}
	\subsection{Documentation}
	The documentation behind QDK is maybe one of the most complete among the various quantum platforms. There are various way to start interacting with the QDK
	\begin{itemize}
		\item Tutorial on their website.
		\item Dedicated Github page.
	\end{itemize}
	The website explain very precisely, with a step by step tutorial, how to install all the mandatory tools in order to use properly their platform. The installation of the required tools is done through .NET platform and directly from Microsoft website.
	 Also on their website we will find a very complete reference of their new programming language completely dedicated to quantum computing, Q\# and various examples of basic quantum circuits.
	 On the other hand on the Github we will find more complex examples which teach us how to use properly Q\# and its integration in the C\# ecosystem.
	
	\subsection{Features}
	\subsubsection{Q\#}	
	The most important feature of QDK is its dedicated programming language Q\#. 
	The model Microsoft is pursuing for quantum computation is to treat the quantum computer as a coprocessor, similar to that used for GPUs, FPGAs. The primary control logic runs classical code on a classical "host" computer. When appropriate and necessary, the host program can invoke a subroutine that runs on the adjunct processor. When the subroutine completes, the host program gets access to the subroutine's results.
	Q\# is a domain-specific programming language used for expressing quantum algorithms. It is to be used for writing subroutines that execute on an adjunct quantum processor, under the control of a classical host program and computer running a classical driver written in C\#.
	
	The syntax of Q\# is rather different from the language of the other platform. It closely resembles C\# and is more verbose than Python. It's key concept is the "operation" construct. This is a callable routine, which is invoked by the C\# driver, with quantum operations. The "operation" is like a function in a classical programming language but manipulating quantum gate, so creating a new operation is like building a new quantum gate. This fact imposes to the operation the same restrictions which are applied to quantum gates. This direct transposition enables the birth of new construct which are more close to the actual representation of quantum circuit. One example could be the keywords: adj(adjointed) and Ctl(controlled) which permit, if an operation is unitary, to define its behaviour when reversed or controlled by a quantum register.
	It's also worth mentioning that Q\# has a type model but permits to let the compiler infers the type of the newly initialised variables.

	\subsubsection{Libraries}
	QDK environment gives us a very rich ecosystem of libraries, which implement the vast majority of basic gate and algorithm currently known. This library list grows day after day allowing developer to have ready to use operator and speeding up development. Some notable mention are:
	\begin{itemize}
	\item \textbf{Microsoft.Quantum.diagnostic} library allows developer to build test in order to check the state of the machine, when possible. Its structure resembles the one of JUnit, with an assert-like structure.
	\item \textbf{Microsoft.Quantum.Simulation.Simulator} gives access to a wide variety of simulators with different property. This allows developer to use specialised simulator for some applications. This topic will be depth in following sections.
	\end{itemize}
	
	\subsubsection{Visual Studio extension}
	Another great feature is the ability to use QDK in the Visual Studio environment. This enable us to manage in a more precise way the interactions and the execution of the code. The integration in Visual Studio also allows developer to have a more precise control over the QDK libraries. Another great features is the debugging ability of VS.
	
	\subsection{Simulator}
	
	
	\subsection{Hardware}
	Unlike the superconducting qubit technology of Rigetti
and IBM, Microsoft is betting highly on topological
qubits based on Majorana fermions. These particles
have recently been discovered and promise long coherence times and other desirable properties, but no functional quantum computer using topological qubits currently exists. So all Q\# applications can't be executed on physical machine exploiting this new technology, but can be executed, like Qiskit, on IBMQuantumExperience by simply changing the driver.


\end{document}

\newpage
